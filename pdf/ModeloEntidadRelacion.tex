\documentclass[11pt,letterpaper]{article}
\usepackage[utf8]{inputenc}
\usepackage[spanish]{babel}
\usepackage[margin=1in]{geometry}
 
\usepackage{listings}
\usepackage{color}
\definecolor{codegreen}{rgb}{0,0.6,0}
\definecolor{codegray}{rgb}{0.5,0.5,0.5}
\definecolor{codepurple}{rgb}{0.58,0,0.82}
\definecolor{backcolour}{rgb}{0.95,0.95,0.92}

\lstdefinestyle{mystyle}{
    backgroundcolor=\color{backcolour},   
    commentstyle=\color{codegreen},
    keywordstyle=\color{magenta},
    numberstyle=\tiny\color{codegray},
    stringstyle=\color{codepurple},
    basicstyle=\footnotesize,
    breakatwhitespace=false,         
    breaklines=true,                 
    captionpos=b,                    
    keepspaces=true,                 
    numbers=left,                    
    numbersep=5pt,                  
    showspaces=false,                
    showstringspaces=false,
    showtabs=false,                  
    tabsize=2
}
\lstset{
style=mystyle,
literate={á}{{\'a}}1
        {ã}{{\~a}}1
        {é}{{\'e}}1
        {ó}{{\'o}}1
        {í}{{\'i}}1
        {ñ}{{\~n}}1
        {¡}{{!`}}1
        {¿}{{?`}}1
        {ú}{{\'u}}1
        {Í}{{\'I}}1
        {Ó}{{\'O}}1
}

\usepackage{graphicx}
\usepackage{enumerate}
\usepackage{enumitem}

\usepackage{longtable}
\usepackage{hyperref}
\usepackage{commath}

\begin{document}

\title{\vspace{-1.5cm}
    Proyecto Final\\
    Tacoste\\
    \large Universidad Nacional Autónoma de México\\
    Facultad de Ciencias\\
    Fundamentos de Bases de Datos\\}
\author{
	Ángel Iván Gladín García\\
    No. cuenta: 313112470\\
    \texttt{angelgladin@ciencias.unam.mx}
    \and
    Marco Antonio Hurtado Gutierrez\\
    No. cuenta: 313110902\\
    \texttt{markhg@ciencias.unam.mx}
    \and
    Luis Fernando Yang Fong Baeza\\
    No. cuenta: 313320679\\
    \texttt{fernandofong@ciencias.unam.mx}
    \and
    María Fernanda González Chávez\\
    No. cuenta: 313036367\\
    \texttt{fernandagch@ciencias.unam.mx}
}
\date{10 de enero de 2018}
\maketitle
%%%%%%%%%%%%%%%%%%%%%%%%%%%%%%%%%%%%%%%%%%%%%%%%%%%%%%%%%%%%%%%%%%%%



%%%%%%%%%%%%%%%%%%%%%%%%%%%%%%%%%%%%%%%%%%%%%%%%%%%%%%%%%%%%%%%%%%%%
\section*{Modelo Entidad Relación}%%%%%%%%%%%%%%%%%%%%%%%%%%%%%%%%%%%%

El objetivo de este modelo es representar de manera grafica la estructura lógica de una base de datos y en este documento se explicara de manera breve como se modelo y porque se modelo de esta manera.


\begin{itemize}

\item Se tiene la entidad sucursal para que podamos guardar a detalle la informacion de cada una de las sucursales, empezando por un identificador            para cada sucursal  su direccion completa y que empleado es el que dirige esta sucursal entonces tenemos toda la informacion relevante.

\item Tenemos la entidad Persona que a su vez tiene dos entidades hijas que son Comensal y Empleado que contiene los datos pertinentes para modelar a       cada una de la mejor manera.

\item Tenemos en la entidad Empleado que se relaciona con la entidad Sucursal a traves de la relacion Trabajar, donde indicamos a que sucursal pertenece cada empleado. Tambien tenemos dos relaciones adicionales: una indica que cada sucursal tiene alguien que la supervisa, y la otra que cada empleado es supervisado por otro (el gerente de la sucursal).

\item La relacion ordenar esta ligada a la entidad comensal y a la entidad pedido para que pueda ordenar lo que el comnesal apetezca.  

\item Un grupo de comensales tienen una mesa donde cada mesa tiene un identificador y ademas cuando los comensales consumen algun producto al final se les entrega un ticket. Se hace le hace el cargo de la cuenta a un solo comensal de la mesa.

\item A cada platillo (producto) se le asocia una salsa. A su vez la salsa tiene una presentacion y nivel de picor.

\item Tambien es importante indicar los ingredientes que se necesitan para la elaboracion de los platillos.

\item Modelamos el registro historico  de los precios y productos ya que nos interesa saber si puede haber una promocion. Posteriormente en la                 base se utilizaran disparadores.

\item Tambien esta la entidad Provedor donde guardamos los datos de los provedores para saber quien suministra a cada sucursal. Con esto obtenemos un  inventario para cada sucursal,        

\item Es importante mencionar que la entidad empleado tiene una entidad hija llamada repartidor la cual es una entidad debil debido a que no tiene llave primaria. Cada repartidor tiene un medio de transporte. Y cada repartidor va a contar con su licencia para poder circular para que no existan problemas legales y esto implica que cada repartidor debe ser mayor de edad.

\end{itemize}




%Sucursal - Contener.
%%%%%%%%%%%%%%%%%%%%%%%%%%%%%%%%%%%%%%%%%%%%%%%%%%%%%%%%%%%%%%%%%%%%
%\begin{thebibliography}{}
%\bibitem{} 
%Google\\
%\url{https://www.google.com.mx/}

%\end{thebibliography}
%%%%%%%%%%%%%%%%%%%%%%%%%%%%%%%%%%%%%%%%%%%%%%%%%%%%%%%%%%%%%%%%%%%%

\end{document}

